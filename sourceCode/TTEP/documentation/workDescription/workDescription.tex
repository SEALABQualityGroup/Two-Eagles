%%This is a very basic article template.
%%There is just one section and two subsections.
\documentclass[12pt]{article}
\usepackage[latin1]{inputenc}
\usepackage[T1]{fontenc}
\usepackage{graphicx}
\usepackage[italian]{babel}
\usepackage[pointlessenum]{paralist}

\title{\textbf{Descrizione dello sviluppo del plugin}}
\date{}
\author{}

\begin{document}

\maketitle

\section{Introduzione}
Questo documento non � necessario per utilizzare il plugin TTEP e relativo help, n� � parte integrante della documentazione di sviluppo. Invece ha come obiettivo la descrizione delle fasi principali dello sviluppo dei plugin, i tool utilizzati e l'ambiente di lavoro in maniera sintetica e informale.

\section{Hardware/software}
Plugin realizzato su:
\begin{itemize}
 \item pc con 768 MB di ram, CPU centrino 1,5 Ghz
 \item Linux (con test di compatibilit� anche su Windows Xp)
\end{itemize}

Sistema di versionamento del progetto:
\begin{itemize}
 \item automatico
 \item utilizzando svn con repository remoto fornito da Assembla.com, indirizzo del progetto \\ http://www.assembla.com/wiki/show/b5O0SY0smr3BFzab7jnrAJ , indirizzo del repository svn \\ http://subversion.assembla.com/svn/ttep (� quindi possibile vedere come si � evouto lo sviluppo del codice essendo conservate tutte le versioni)
 \item il progetto � diventato pubblico in corso d'opera per cambiamento delle politiche di fornitura del servizio del sito
\item nel progetto pubblico NON � (e non � mai stato) compreso codice di TwoTowers o comunque codice non inerente il plugin stesso
\end{itemize}

Versioni di Eclipse utilizzate:
\begin{itemize}
 \item inizio del lavoro su Eclipse Europa
 \item passaggio in corso d'opera ad Eclipse Ganymede
 \item la versione pi� recente, Galileo, non � stata utilizzata nello sviluppo
 \item update costanti e tuning del file di configurazione (altrimenti frequenti crash di Eclipse)
\end{itemize}

Tool di supporto usati:
\begin{itemize}
 \item Plugin di Eclipse per:
\begin{itemize}
 \item modelli uml (Omondo -> eUml2 -> Umlet)
 \item latex (Texlipse)
 \item svn (Subversive)
\end{itemize}

\end{itemize}

\section{Sviluppo del progetto}

Fonti di informazioni maggiormente utilizzate:
\begin{itemize}
 \item documentazione ufficiale di Eclipse
 \item paper, tutorial, documenti presenti nel sito ufficiale di Eclipse
 \item mailing list specifiche per lo sviluppo di plugin Eclipse
 \item Google code (code.google.com)
 \item vari siti internet che trattano lo sviluppo di plugin Eclipse.
\end{itemize}

Lunga fase di studio di Eclipse, sia iniziale, sia durante lo sviluppo. Da utente:
\begin{itemize}
 \item comprensione della terminologia (es. prospettiva, vista, workspace, workbench...)
 \item logiche di funzionamento del programma
 \item padronanza della gestione del tool
\end{itemize}
Da sviluppatore:
\begin{itemize}
 \item studio dell'architettura generale
 \item comprensione della logica dei plugin
 \item individuazione delle componenti dell'architettura su cui intervenire per raggiungere l'obiettivo
 \item scelta (non sempre a priori) tra diverse metodologie di implementazione di funzionalit� (es. men� actionSet vs command)
 \item controllo dell'evoluzione della piattaforma Eclipse al fine di non produrre un plugin obsoleto ancor prima del rilascio (es. evitato l'utilizzo di classi/metodi o pattern deprecati)
\end{itemize}

Studio del tool TwoTowers:
\begin{itemize}
 \item funzionamento generale
 \item architettura generale
 \item logica di funzionamento del codice sorgente
 \item individuazione dei punti di collegamento su cui agganciare il plugin
\end{itemize}

\end{document}
