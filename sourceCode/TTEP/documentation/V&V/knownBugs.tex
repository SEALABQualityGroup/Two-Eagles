%%This is a very basic article template.
%%There is just one section and two subsections.
\documentclass[12pt]{article}
\usepackage[latin1]{inputenc}
\usepackage[T1]{fontenc}
\usepackage{graphicx}
\usepackage[italian]{babel}
\usepackage[pointlessenum]{paralist}

\title{\textbf{Bug e problemi noti del plugin TTEP}}
\date{}
\author{}

\begin{document}

\maketitle
% \section{Bug e problemi noti del plugin TTEP}
\begin{itemize}
 \item Solo su Windows Xp � stato riscontrato il mancato aggiornamento automatico della vista 'Navigator', gestisto direttamente da TTEP, in seguito all'esecuzione di una qualsiasi funzionalit� che produce dei nuovi file in output. Il problema pu� comunque essere risolto abilitando l'impostazione associata al refresh automatico del workspace: Window -> Preferences -> General -> Workspace -> Refresh automatically .

Il metodo che si occupa di ci� � 'refreshNavigator()', nella classe Utility del package it.univaq.ttep.utility .
 \item Su Linux con kernel 2.6.29.4, Fluxbox 1.1.1 come windowmanger, Eclipse 3.4.2, si possono presentare crash dell'intero server grafico in seguito al lancio di alcune funzionalit� (es. Transient probability distribution calculator) plugin TTEP. Non sono presenti anomalie n� nel file di log di Eclipse, n� in quello del server grafico X. Usando come desktop environment Kde 4.2.4 il problema non si verifica. Esente da problemi anche Windows Xp;
 \item durante l'esecuzione della funzionalit� 'Simulator' del men� 'Performance Evaluator' se il calcolo svolto da TTKernel non � istantaneo si ha una discrepanza tra quanto sembra indicare la finestra di dialogo che mostra l'avanzmento dell'operazione e il tempo realmente impiegato per il calcolo. Il metodo che si occupa di ci� � l''execute' chiamato da 'Simulator', nella classe Utility del package it.univaq.ttep.utility .
\end{itemize}





\end{document}
